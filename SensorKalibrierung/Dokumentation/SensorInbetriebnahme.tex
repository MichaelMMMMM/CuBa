\documentclass{article}
\usepackage[ngerman]{babel}
\usepackage[utf8]{inputenc}
\usepackage{graphicx} 
\usepackage{svg}
\graphicspath{{Bilder/}}
\usepackage{geometry}
\geometry{a4paper, top=25mm, left=30mm, right=25mm, bottom=20mm}
\begin{document}

\section{Inbetriebnahme der Sensor}
Die Werte $\phi_K$, $\dot{\phi_K}$ und $\ddot{\phi_K}$ müssen für die Regelung des Würfels erfasst werden. Hierfür wird das Bauteil \textit{2020 Adafruit Industries} verwendet. Dieser Chip verfügt über einen Beschleunigungssensor, ein Gyroskop und Magnetometer, welche jeweils über drei Achsen die Beschleunigung, die Winkelgeschwindigkeit und die magnetische Flussdichte messen. 
Um die Sensoren in das Gesamtsystem einzubinden müssen diese über $I^2C$ mit der Rechnereinheit verbunden und kalibriert werden. Außerdem müssen die gemessenen Werte in die benötigten Größen überführt werden.

\subsection{Anbindung über $I^2C$ und Sensorkonfiguration}
Der Sensor unterstützt das $I^2C$ Protokoll bei einer Frequenz von $400k Hz$. Mit Hilfe der Registeradressen des Sensors, welche im Datenblatt dokumentiert sind, werden die aktuellen Sensorwerte ausgelesen und Konfigurationen festgelegt werden.

Der Beschleunigungssensor erfasst Beschleunigungen im Bereich von $\pm2 g$ mit einer Auflösung von $0.061 mg / LSb$. Die Winkelgeschwindigkeit des Gyroskop bewegt sich in einem Bereich von $\pm 245 dps$ mit einer Auflösung von $8.75 mdps / LSb$. Der Magnetometer liefert Werte im Bereich von $\pm 2gauss$ und bringt hierbei eine Auflösung von $0.08 mgauss / LSb$.

\subsection{Ermittlung der Größen $\phi_K$, $\dot{\phi_K}$ und $\ddot{\phi_K}$}
Die Sensoren erfassen die jeweilige Größe in drei Achsen. Betrachtet man die Drehbewegung der Würfelseite, so entspricht die Y-Achse des Sensors der Radialachse des Würfels und die X-Achse des Sensors der Tangentialachse der Drehbewegung. Die Z-Achse des Sensors steht senkrecht auf der Drehebene.

\subsubsection{Bestimmung von $\phi_K$}
Der Beschleunigungssensor erfasst die drei Beschleunigungswerte in $g$ bzw. in der SI-Einheit $\frac{m}{s^2}$. Aus den X- und Y-Werten kann die aktuelle Winkelbeschleunigung $\ddot{\phi_K}$ bestimmt werden. Hierbei muss beachtet werden, dass der Sensor in Ruhelage eine absolute Beschleunigung von $1g$ ausgibt, also offset-behaftet ist. Die Zusammensetzung der Beschleunigungswerte kann winkelabhängig bestimmt werden.

\[\ \ddot{x_S} := Beschleunigungswert \ in \ X \ in \ \frac{m}{s^2} \]
\[\ \ddot{y_S} := Beschleunigungswert \ in \ Y \ in \ \frac{m}{s^2} \]
\[\ \ddot{z_S} := Beschleunigungswert \ in \ Z \ in \ \frac{m}{s^2} \]
\[\ r_S := Abstand \ von  \ Sensor \ zu \ Drehpunkt \]

\[\ (\ddot{x_S}, \ddot{y_S}, \ddot{z_S}) = (r_S \cdot \ddot{\phi_K} + sin(\phi_K) \cdot g , -r_S \cdot {\dot{\phi_K}}^2 - cos(\phi_K) \cdot g , 0) \]

\[\ \ddot{\phi_K} = \frac{\ddot{x_S} - sin(\phi_K) \cdot g }{r_S} \]

\subsubsection{Bestimmung von $\dot{\phi_K}$}
Das Gyroskop gibt die Winkelgeschwindigkeiten in Grad pro Sekunde wider. Somit muss der Z-Wert lediglich in SI-Einheiten umgewandelt werden.

\[\ {\omega_S} := Winkelgeschwindigkeit \ um \ Z \ in \ Grad \ pro \ Sekunde \]

\[\ \dot{\phi_K} = {\omega_S} \cdot \frac{2 \cdot \pi}{360} \]

\subsubsection{Bestimmung von $\phi_K$}
Der Ausfallwinkel des Würfels kann über verschiedene Wege ermittelt werden. Nach der Kalibrierung des Magnetometer kann über das Verhältnis der magnetischen Flussdichte in X- und Y-Richtung der Winkel berechnet werden. Dieses Verfahren hat sich allerdings als extrem störanfällig bewiesen und ist somit für diesen Anwendungsfall untauglich.

Über die folgende Differenzialgleichung zur Bestimmung des dynamischen Verhalten des Systems kann der Winkel $\phi_K$ berechnet werden.

\[\ (\Theta_K^{(A)} + m_w \cdot l^2) \cdot \ddot{\phi_K} = (m_K  l_K + m_w \cdot l) \cdot g \cdot sin(\phi_K)) - C_K \cdot \dot{\phi_w} - T_M  \]

\[\ sin(\ddot{phi_K}) = (\Theta_K^{(A)} + m_w \cdot l^2) \cdot \ddot{\phi_K} + C_K \cdot \dot{\phi_w} + T_M \]

Die Sensoren liefern die Werte $\ddot{\phi_K}$ und $\dot{\phi_K}$, die Winkelgeschwindigkeit $\dot{\phi_R}$ wird über den Motortreiber ermittelt. Somit kann $\phi_K$ mittels der obigen Gleichung berechnet werden. Allerdings müssen die konstanten Werte bereits bestimmt worden sein. Folglich kann diese Methode erst nach erfolgreicher Systemidentifikation verwendet werden.

Um die Systemidentifikation durchzuführen wird ein zweiter Sensor an der Würfelseite angebracht. Über das Verhältnis der Abstände der Sensoren zum Drehpunkt und deren Beschleunigungswerte kann eine Winkelschätzung durchgeführt werden. Mit diesem Prinzip können die Systemparameter bestimmt werden und anschließend die oben genannte Methode verwendet werden. Dadurch ist der zweite Sensor lediglich zur Systemidentifikation notwendig.

\[\ r_S1 := Abstand \ von \ Sensor \ 1 \ zu \ Drehpunkt \]
\[\ r_S2 := Abstand \ von \ Sensor \ 2 \ zu \ Drehpunkt \]
\[\ \mu = \frac{r_{S1}}{r_{S2}} \]

\[ \ddot{x}_{S1} - \mu \cdot \ddot{x}_{S2} = (1 - \mu) \cdot g \cdot sin(\phi_K) \]

\[\ \ddot{y}_{S1} - \mu \cdot \ddot{y}_{S2} = - (1 - \mu) \cdot g \cdot cos(\phi_K) \]

\[\ \frac{\ddot{x}_{S1} - \mu \cdot \ddot{x}_{S2}}{\ddot{y}_{S1} - \mu \cdot \ddot{y}_{S2}} = 
-tan(\phi_K) \]

\[\ \phi_K = - atan(\frac{\ddot{x}_{S1} - \mu \cdot \ddot{x}_{S2}}{\ddot{y}_{S1} - \mu \cdot \ddot{y}_{S2}}) \]


\subsection{Kalibrierung der Sensoren}
Die erforderlichen Sensorwerte um die benötigten Größen zu bestimmen sind $\ddot{x}_S$, $\ddot{y}_S$ und $\omega_S$. Die Sensoren geben die physikalischen Größen in 16-Bit-Werten in der Zweierkomplement-Darstellung wieder. Über einen Faktor kann der entsprechende Wert als SI-Einheit dargestellt werden. Bei der Kalibrierung muss der Offset zwischen Sollausgabe und Istwert ermittelt werden, um die Sensorwerte korrigieren zu können.

\subsubsection{Kalibrierung des Beschleunigungssensor}
In der Ruhelage soll der Sensor einen Beschleunigungsbetrag von $1g$ anzeigen. Im ersten Aufbau wird der Sensor so positioniert, dass die Gravitation lediglich den X-Wert beeinflusst, folglich ist der Sollwert der X-Beschleunigung $1g$. Um die mittlere Abweichung $\ddot{x}_{off}$ zu ermitteln wird eine Messreihe von $m = 10000$ Messwerten aufgezeichnet. 

\[\ \ddot{x}_{off} = 1g - \frac{\sum_{i = 1}^{m}}{m} \ddot{x} \]

Die oben berechnete Abweichung bezieht sich auf die Darstellung des Wertes als SI-Einheit. Um den Offset in Ganzzahldarstellung zu ermitteln muss durch den Umrechnungsfaktor dividiert und gerundet werden.

\[\ \ddot{x}_{off}^{int} = \frac{\ddot{x}_{off}}{factor} \]

Die Bestimmung der Abweichung der Beschleunigung für die Y-Achse verläuft analog.

\subsubsection{Kalibrierung des Gyroskop}
Der Sollwert des Gyroskop in der Ruhelage beträgt $0 dps$. Somit lässt sich der Offset analog zu dem Beschleunigungssensor ermitteln.

\subsubsection{Messergebnis für X-Achse des Beschleunigungssensor}
Die Messung zur Bestimmung des Offset der X-Achse wurde zweimal durchgeführt. Zuerst wurde die X-Achse des Sensor in Gravitationsrichtung gestellt. Bei dem zweiten Durchlauf zeigte sie entgegen der Gravitationsrichtung.

\[\ Messung \ X-Achse \ Beschleunigungssensor \ in \ Gravitationsrichtung \]
\[\ Durchgefuehrt \ am \ 10.05.2016 \]
\[\ Anzahl \ Messungen : m = 10000 \]
\[\ \frac{\sum_{i = 1}^{m} \ddot{x}^{int}}{m} = -19770 \]
\[\ \frac{\sum_{i = 1}^{m} \ddot{x}}{m} = -1.0966 \]
\[\ \ddot{x}_{off}^{int} = 34371 \]
\[\ \ddot{x}_{off} = 2.0966 \]

\[\ Messung \ X-Achse \ Beschleunigungssensor \ gegen \ Gravitationsrichtung \]
\[\ Durchgefuehrt \ am \ 10.05.2016 \]
\[\ Anzahl \ Messungen : m = 10000 \]
\[\ \frac{\sum_{i = 1}^{m} \ddot{x}^{int}}{m} = 15057 \]
\[\ \frac{\sum_{i = 1}^{m} \ddot{x}}{m} = 0.9185 \]
\[\ \ddot{x}_{off}^{int} = 1336 \]
\[\ \ddot{x}_{off} = 0.0815 \]

Die Ergebnisse zeigen, dass die Achsen auf dem Chip falsch markiert sein. Die Richtung kann allerdings problemlos korrigiert werden.

\subsubsection{Messergebnis für Y-Achse des Beschleunigungssensor}
Die Messung wurde analog zu der X-Achse zweimal durchgeführt. Die aufgezeichnete Sensorachse ist ebenfalls invertiert.

\[\ Messung \ Y-Achse \ Beschleunigungssensor \ in \ Gravitationsrichtung \]
\[\ Durchgefuehrt \ am \ 10.05.2016 \]
\[\ Anzahl \ Messungen : m = 10000 \]
\[\ \frac{\sum_{i = 1}^{m} \ddot{y}^{int}}{m} = -16085 \]
\[\ \frac{\sum_{i = 1}^{m} \ddot{y}}{m} = -0.9812 \]
\[\ \ddot{y}_{off}^{int} = 32479 \]
\[\ \ddot{y}_{off} = 1.9812 \]


\[\ Messung \ Y-Achse \ Beschleunigungssensor \ gegen \ Gravitationsrichtung \]
\[\ Durchgefuehrt \ am \ 10.05.2016 \]
\[\ Anzahl \ Messungen : m = 10000 \]
\[\ \frac{\sum_{i = 1}^{m} \ddot{y}^{int}}{m} = 16293 \]
\[\ \frac{\sum_{i = 1}^{m} \ddot{y}}{m} = 0.9939 \]
\[\ \ddot{y}_{off}^{int} = 100 \]
\[\ \ddot{y}_{off} = 0.0061 \]

\subsubsection{Messergebnis für die Z-Achse des Gyroskop}
Die Messung zur Bestimmung des Offset der Z-Achse wurde einmal durchgeführt.

\[\ Messung \ Z-Achse \ Gyroskop\]
\[\ Durchgefuehrt \ am \ 10.05.2016 \]
\[\ Anzahl \ Messungen : m = 10000 \]
\[\ \frac{\sum_{i = 1}^{m} {\omega}^{int}}{m} = 55 \]
\[\ \frac{\sum_{i = 1}^{m} {\omega}^{dps}}{m} = 0.4793 \]
\[\ {\omega}_{off}^{int} = -55 \]
\[\ {\omega}_{off} = -0.4793 \]

\end{document}